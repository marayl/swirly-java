\documentclass[12pt,a4paper,notitlepage,bibliography=totoc]{scrreprt}

\usepackage{hyperref}
\hypersetup{
  bookmarks=true,
  pdftitle={Swirly Cloud},
  pdfauthor={Mark Aylett},
  colorlinks=true,
  linkcolor=blue,
  citecolor=blue,
  filecolor=blue,
  urlcolor=blue
}

\usepackage{draftwatermark}
\SetWatermarkLightness{0.9}
\SetWatermarkText{Confidential}
\SetWatermarkScale{3}

\usepackage{float}
\usepackage{graphicx}
\usepackage{listings}
\usepackage[titletoc,title,toc]{appendix}
\usepackage[acronym,toc,xindy]{glossaries}
\makeglossaries

\newcommand{\ceil}[1]{\left\lceil #1 \right\rceil}
\newcommand{\floor}[1]{\left\lfloor #1 \right\rfloor}

\newcommand{\swirly}{Swirly Cloud}
\newcommand{\s}{\textsterling}

\newacronym{api}{API}{Application Programming Interface}
\newacronym{bbo}{BBO}{Best Bid and Offer}
\newacronym{clob}{CLOB}{Central Limit Order-Book}
\newacronym{cpu}{CPU}{Central Processing Unit}
\newacronym{ecn}{ECN}{Electronic Communication Network}
\newacronym{fok}{FOK}{Fill Or Kill}
\newacronym{gtc}{GTC}{Good `Till Cancelled}
\newacronym{json}{JSON}{Java-Script Object Notation}
\newacronym{hft}{HFT}{High-Frequency Trading}
\newacronym{http}{HTTP}{Hyper-Text Transfer Protocol}
\newacronym{ip}{IP}{Intellectual Property}
\newacronym{mis}{MIS}{Market Information Services}
\newacronym{mtm}{MTM}{Mark To Market}
\newacronym{otc}{OTC}{Over The Counter}
\newacronym{poc}{POC}{Proof Of Concept}
\newacronym{rca}{RCA}{Root Cause Analysis}
\newacronym{rest}{REST}{REpresentational State Transfer}
\newacronym{tco}{TCO}{Total Cost of Ownership}
\newacronym{ttl}{TTL}{Time To Live}
\newacronym{tvm}{TVM}{Time Value of Money}
\newacronym{tob}{TOB}{Top Of Book}
\newacronym{ui}{UI}{User Interface}
\newacronym{vwap}{VWAP}{Volume Weighted Average Price}
\newacronym{www}{WWW}{World Wide Web}

\newglossaryentry{appengine}{name={App Engine}, description={application services running on
    Google's Cloud platform}}

\newglossaryentry{arbitrage}{name={arbitrage}, description={the simultaneous purchase and sale on
    different markets, of the same or equivalent financial instruments to profit from price or
    currency differentials}}

\newglossaryentry{asset}{name={asset}, description={an item of value}}

\newglossaryentry{auction}{name={auction}, description={a market where goods or services are sold to
    the highest bidder}}

\newglossaryentry{bid}{name={bid}, description={the price and quantity at which a market participant
    is willing to buy}}

\newglossaryentry{cloud}{name={Cloud}, description={a network of remote servers hosted on the
    Internet to store, manage, and process data}}

\newglossaryentry{contract}{name={contract}, description={a specification that stipulates the terms
    and conditions of sale}}

\newglossaryentry{e-commerce}{name={e-commerce}, description={commercial transactions conducted
    electronically on the Internet.}}

\newglossaryentry{exchange}{name={exchange}, description={a place where buyers and sellers meet to
    exchange goods or services}}

\newglossaryentry{execution}{name={execution}, description={a transaction that occurs as an order
    transitions through a workflow}}

\newglossaryentry{expiry date}{name={expiry date}, description={the last date that a contract can be
    traded on a specific market}}

\newglossaryentry{given}{name={given}, description={when the taker hits the bid}}

\newglossaryentry{price-level}{name={price-level}, description={a price level is the sum of all
    orders in the book at the same price}}

\newglossaryentry{limit order}{name={limit order}, description={an order to buy or sell at a
    specific price or better}}

\newglossaryentry{limit price}{name={limit price}, description={the price on a limit order}}

\newglossaryentry{liquid market}{name={liquid market}, description={the ability of a market to
    accept large transactions with minimal impact on price stability}}

\newglossaryentry{liquidity}{name={liquidity}, description={refers the bids and offers available in
    a market}}

\newglossaryentry{lot}{name={lot}, description={a unit of measure that represents the smallest
    amount that can be bought or sold}}

\newglossaryentry{maker}{name={maker}, description={passive buyer or seller that receives the spread}}

\newglossaryentry{market}{name={market}, description={a place where buyers and sellers come together
    to exchange goods or services}}

\newglossaryentry{market-maker}{name={market-maker}, description={a market-maker quotes both bids
    and offers, aiming to profit from the bid-offer spread}}

\newglossaryentry{market order}{name={market order}, description={an order to buy or sell at the
    current market price}}

\newglossaryentry{market-place}{name={market-place}, description={a venue comprising many markets}}

\newglossaryentry{matching engine}{name={matching engine}, description={the software component in a
    trading application responsible for matching buy and sell orders to form a trade}}

\newglossaryentry{merchant}{name={merchant}, description={a person or company involved in wholesale
    trade, especially one dealing with foreign countries or supplying goods to a particular trade.}}

\newglossaryentry{order-book}{name={order-book}, description={the software component in a trading
    application responsible for managing resting orders in a two-sided market}}

\newglossaryentry{offer}{name={offer}, description={the price and quantity at which a market
    partipant is willing to sell}}

\newglossaryentry{order}{name={order}, description={an instruction to buy or sell goods or
    services}}

\newglossaryentry{order-driven}{name={order-driven}, description={a financial market where the
    orders for both buyers and sellers are displayed along with their price and quantity}}

\newglossaryentry{paid}{name={paid}, description={when the taker lifts the offer}}

\newglossaryentry{quote}{name={quote}, description={the price offered by a market-maker to buy or
    sell a specific quantity of goods}}

\newglossaryentry{quote-driven}{name={quote-driven}, description={a financial market where
    market-makers send quotes to interested parties that guarantee order execution.}}

\newglossaryentry{restful}{name={restful}, description={a software architecture for building
    scalable web services}}

\newglossaryentry{resting-order}{name={resting-order}, description={an order that is sitting in the
    order-book waiting to be matched}}

\newglossaryentry{settlement date}{name={settlement date}, description={the date on which a trade is
    settled between counter-parties}}

\newglossaryentry{spread}{name={spread}, description={the difference between the best bid and
    offer}}

\newglossaryentry{taker}{name={taker}, description={aggressive buyer or seller that pays the spread}}

\newglossaryentry{tick}{name={tick}, description={a unit of measure that represents the smallest
    price movement}}

\newglossaryentry{trade}{name={trade}, description={the exchange of goods or services between
    counter-parties}}

\newglossaryentry{trade date}{name={trade date}, description={the date on which a trade takes place}}

\newglossaryentry{volume}{name={volume}, description={total quantity}}

\newglossaryentry{web}{name={Web}, description={the World Wide Web}}

\glsaddall

\begin{document}

\title{Swirly Cloud}
\subtitle{scalable and transparent quoting}
\author{Mark Aylett}
\date{\today\\Version 0.1}
\maketitle

\begin{abstract}

In this paper, we will argue that \gls{order-driven} systems are better suited to modern financial
\glspl{market} than their \gls{quote-driven} counterparts, while acknowledging that
\gls{quote-driven} \glspl{market} still play an important role in guaranteeing \gls{order} execution
in the retail sector. We then show how \swirly{} combines the best of both systems to deliver
scalable and transparent retail quoting.

\end{abstract}

\tableofcontents

\chapter{Introduction}

In this chapter, we will see how many of the key concepts used by large financial \glspl{exchange},
apply equally well to more familiar experiences with online retailers, \glspl{auction}, and
price comparison websites. This introduction will also serve to orient the reader with the
nomenclature used later in this document.

We have chosen to use coffee beans in our introductory examples, because coffee is an important
commodity, and a great way to get started!

\section{Online Retailer}

A retail consumer in the UK might decide to purchase coffee beans direct from an online
retailer. Suppose that an online retailer called ``eMarket'' in selling coffee beans at \s10.00 per
bag, and they currently have 12 bags in stock. This can be depicted as follows:

\vspace{5mm}
\begin{tabular}{rr}
\multicolumn{2}{c}{eMarket Coffee Beans}\\
Offer Price&Offer Lots\\
\hline
\texttt{10.00}&\texttt{12}\\
\end{tabular}
\vspace{5mm}

Each ``\gls{lot}'' in this example represents a large bag of Coffee Beans. (Note that we have yet to
specify the exact type and quality of the beans.) If we decide to buy 5 of the 12 \glspl{lot} on
``\gls{offer}'', then only 7 will remain. The online retailer may then, perhaps questionably, decide
to raise the price if they believe that their supply is insufficient to meet demand:

\vspace{5mm}
\begin{tabular}{rr}
\multicolumn{2}{c}{eMarket Coffee Beans}\\
Offer Price&Offer Lots\\
\hline
\texttt{10.50}&\texttt{7}\\
\end{tabular}
\vspace{5mm}

\section{Online Auction}

The price is offered by the online retailer is presumably non-negotiable, so we have to either
``take'' it or leave it. But what if we could negotiate, or ``\gls{bid}'' for the remaining
\glspl{lot}? You would then have a system similar to an online ``\gls{auction}'', such as
eBay\cite{ebay}, where the \gls{offer} of \s10.50 is effectively the ``Buy Now'' price. Let's say we
place a \gls{bid} to buy 3 more \glspl{lot} of eMarket Coffee Beans at \s10.20:

\vspace{5mm}
\begin{tabular}{rrrrr}
\multicolumn{5}{c}{eMarket Coffee Beans}\\
Bid Lots&Bid Price&Spread&Offer Price&Offer Lots\\
\hline
\texttt{3}&\texttt{10.20}&\texttt{0.30}&\texttt{10.50}&\texttt{7}\\
\end{tabular}
\vspace{5mm}

There are now two sides of the \gls{market}: the \gls{bid}- and \gls{offer}-side. The difference
between the \gls{bid} and \gls{offer} price is 30 pence, which is known as the ``\gls{spread}''. In
an online \gls{auction}, another buyer may compete by raising the ``best'' \gls{bid} to ``buy 2
\glspl{lot} at \s10.30'':

\vspace{5mm}
\begin{tabular}{r|rrrrr}
\multicolumn{6}{c}{eMarket Coffee Beans}\\
Level&Bid Lots&Bid Price&Spread&Offer Price&Offer Lots\\
\hline
\texttt{1}&\texttt{2}&\texttt{10.30}&\texttt{0.20}&\texttt{10.50}&\texttt{7}\\
\texttt{2}&\texttt{3}&\texttt{10.20}&\texttt{-}&\texttt{-}&\texttt{-}\\
\end{tabular}
\vspace{5mm}

The \gls{bid}-side of the \gls{market} now has two ``\glspl{price-level}'', while the
\gls{offer}-side has just one. A collection of \glspl{order} organised by side and \gls{price-level}
in this way is known as an ``\gls{order-book}''. The best \gls{price-level} on each side of the
\gls{order-book} is \gls{tob}.

Our \gls{order} is said to be ``resting'' at the second \gls{price-level}, beneath \gls{tob}. The
\gls{spread} on the first \gls{price-level} has now been reduced to 20 pence. (There is no
\gls{spread} on the second level, because there is no \gls{offer} at this level.)

Note also at this point that the average price the seller can expect to receive for 5 of the 7
\glspl{lot} they are offering is \s10.24:

\[
\frac{1}{n}\sum_{i=1}^{n}p_iq_i = \frac{2\times10.30+3\times10.20}{5} = 10.24
\]

Where $p$ is price, $q$ is \glspl{lot}, and $n$ is total \glspl{lot} (or ``\gls{volume}''). This is
known as the \gls{vwap}.

If this new buyer then decides to ``revise'' their \gls{bid} with a more ``aggressive'' price of
\s10.50, then their \gls{bid} will ``cross the \gls{spread}'' and match the \gls{offer} to create a
trade, which will reduce the remaining supply down to 5 \glspl{lot}. The buyer effectively decided
to ``buy now''. The \gls{offer} of \s10.20 will once again rise to be \gls{tob}:

\vspace{5mm}
\begin{tabular}{r|rrrrr}
\multicolumn{6}{c}{eMarket Coffee Beans}\\
Level&Bid Lots&Bid Price&Spread&Offer Price&Offer Lots\\
\hline
\texttt{1}&\texttt{3}&\texttt{10.20}&\texttt{0.30}&\texttt{10.50}&\texttt{5}\\
\texttt{2}&\texttt{-}&\texttt{-}&\texttt{-}&\texttt{-}&\texttt{-}\\
\end{tabular}
\vspace{5mm}

All manner of goods can be sold in an online \gls{auction}. Oftentimes, the items on sale are one of
a kind. If would be difficult, for example, to \gls{offer} many second-hand bicycles of the exact
same specification, so each would have to be sold separately in its own \gls{market}.

\section{Price Comparison}

It is interesting to contrast the idea of an online \gls{auction} with a price-comparison
website. Perhaps, as prospective buyers, we're interested in finding the best price for the latest
smart phone:

\vspace{5mm}
\begin{tabular}{r|rrrrr}
\multicolumn{6}{c}{Smart Phone Model XYZ}\\
Level&Bid Lots&Bid Price&Spread&Offer Price&Offer Lots\\
\hline
\texttt{1}&\texttt{-}&\texttt{-}&\texttt{-}&\texttt{419.99}&\texttt{5}\\
\texttt{2}&\texttt{-}&\texttt{-}&\texttt{-}&\texttt{429.99}&\texttt{12}\\
\texttt{3}&\texttt{-}&\texttt{-}&\texttt{-}&\texttt{449.99}&\texttt{20}\\
\end{tabular}
\vspace{5mm}

Here there are 3 \glspl{price-level} in this \gls{order-book}, and we can see that, unsurprisingly,
more smart phones are available at a higher prices. An \gls{order-book} is slightly different to a
price comparison website, however, because an \gls{order-book} aggregates \glspl{order}
(\glspl{offer} in this case) at each \gls{price-level}, rather than by each supplier. The 20
\glspl{lot} available at the third \gls{price-level}, for example, may comprise \glspl{order} from
many suppliers, so attempting to take all 20 \glspl{lot} may result in trades with more than one
supplier.

We could use this same system to compare prices from coffee bean suppliers, but we'd need to be more
specific about the exact type and quality of the beans, and how and when they would be
delivered. This information would form the basis of a ``\gls{contract}'':

\vspace{5mm}
\begin{tabular}{r|rrrrr}
\multicolumn{6}{c}{Aribica Coffee Beans per Kilo}\\
\multicolumn{6}{c}{September 2015}\\
Level&Bid Lots&Bid Price&Spread&Offer Price&Offer Lots\\
\hline
\texttt{1}&\texttt{-}&\texttt{-}&\texttt{-}&\texttt{10.50}&\texttt{5}\\
\texttt{2}&\texttt{-}&\texttt{-}&\texttt{-}&\texttt{10.60}&\texttt{12}\\
\texttt{3}&\texttt{-}&\texttt{-}&\texttt{-}&\texttt{10.80}&\texttt{25}\\
\end{tabular}
\vspace{5mm}

\section{Wholesale Market}

If we combine the notion of an online \gls{auction} with a price comparison website, then we end up
with something much more reminiscent of a wholesale \gls{market} on an \gls{exchange}, where there
are many active buyers and sellers:

\vspace{5mm}
\begin{tabular}{r|rrrrr}
\multicolumn{6}{c}{Aribica Coffee Beans per Kilo}\\
\multicolumn{6}{c}{September 2015}\\
Level&Bid Lots&Bid Price&Spread&Offer Price&Offer Lots\\
\hline
\texttt{1}&\texttt{10}&\texttt{10.40}&\texttt{0.10}&\texttt{10.50}&\texttt{5}\\
\texttt{2}&\texttt{15}&\texttt{10.25}&\texttt{0.35}&\texttt{10.60}&\texttt{12}\\
\texttt{3}&\texttt{-}&\texttt{-}&\texttt{-}&\texttt{10.80}&\texttt{25}\\
\end{tabular}
\vspace{5mm}

A larger number of \gls{market} participants generally helps to improve the \gls{market}'s
efficiency and stability. This is known as a ``\gls{liquid market}''. \Gls{market} participants
would typically enter electronic \glspl{order} via \gls{ui} similar to the following:

\begin{figure}[H]
\centering
\includegraphics[scale=0.6]{order-entry.eps}
\caption{Order Entry}
\end{figure}

Financial organisations use such systems to trade a wide range of products spanning different
\gls{asset}-classes.

In this chapter, we've seen how \glspl{order-book} can represent different kinds of
\gls{market-place}, including online \glspl{auction}, price comparison websites, and wholesale
\glspl{market}.

\chapter{Market Mechanics}

\section{Quote Driven Markets}

A \gls{quote-driven} \gls{market} is a financial \gls{market} where \glspl{market-maker} send
\glspl{quote} to interested parties that guarantee order execution.

\begin{figure}[H]
\centering
\includegraphics[scale=0.6]{quote-model.eps}
\caption{Quote Model}
\end{figure}

\section{Order Driven Markets}

An \gls{order-driven} \gls{market} is a financial \gls{market} where the \glspl{order} for both
buyers and sellers are displayed along with their price and quantity.

\begin{figure}[H]
\centering
\includegraphics[scale=0.6]{order-model.eps}
\caption{Order Model}
\end{figure}

\chapter{Swirly Cloud}

\swirly{} is a retail quoting-engine backed by an \gls{order-driven} wholesale \gls{exchange}. By
combining the generality of \glspl{order-book} with the pervasiveness of mobile and \gls{web}
platforms, \swirly{} benefits the entire value-chain from retail \gls{e-commerce} \glspl{merchant}
to wholesale \glspl{market-maker}.

\section{Overview}

\Glspl{market-maker} generate \gls{liquidity} by placing \gls{gtc} \glspl{order} into \swirly{}'s
\gls{clob}. \swirly{}'s Quoting-Engine then uses this \gls{liquidity} to generate two-sided
\glspl{quote} for retail customers. (A two-sided \gls{quote} is a \gls{quote} with a \gls{bid} and
\gls{offer} price.) \swirly{}'s system for managing retail \glspl{quote} on behalf of wholesale
\glspl{market-maker} leads to more efficient, fair and transparent \glspl{market}, which are
essential for regulatory compliance.

\begin{figure}[H]
\centering
\includegraphics[scale=0.6]{hybrid-model.eps}
\caption{Hybrid Model}
\end{figure}

\section{Retail Quotes}

\Glspl{quote} are generated in response to \gls{quote} requests sent by the customer for a specific
\gls{market} and quantity. \Glspl{quote} have a \gls{ttl}, after which they are implicitly cancelled
by \swirly{}. A customer may choose to \gls{trade} on a \gls{quote}, by submitting a ``previously
quoted'' \gls{order} to \swirly{}'s \gls{matching engine}. These previously quoted \glspl{order} are
similar to \gls{fok} \glspl{order}, in that they are either fully filled or Cancelled
immediately. Specifically, previously quoted \glspl{order} cannot be partially filled, and they are
further guaranteed to be filled by exactly one \gls{market-maker}, which simplifies the retail
settlement process.

\section{Order Cancellation}

An \gls{order} is said to be ``Quoted'' when some of its residual quantity is referenced by one or
more unexpired \glspl{quote}. If the \gls{market-maker} attempts to Cancel a quoted \gls{order},
then the \gls{order} will be marked as Pending-Cancel until all of its \glspl{quote} have Expired,
at which point, the \gls{order} will be Cancelled. Although Pending-Cancel \glspl{order} are
withdrawn from the \gls{clob} immediately, they may still Execute with existing \glspl{quote} for up
to \gls{ttl} seconds (until all \glspl{quote} have Expired). \Glspl{market-maker} are expected to
compensate for this in their \gls{limit order} prices.

\section{Quoting Model}

\swirly{} \glspl{order} have an ``Available'' quantity in addition to the quantities normally
associated with \gls{order-driven} \glspl{market}, such as \gls{order}, Residual, Executed and
Last-Traded. The Available quantity is used to track the sum of all unexpired \glspl{quote}
associated with an \gls{order}.

When the \gls{matching engine} receives a \gls{quote} Request, it searches each side of the
\gls{order-book} in Price-Time priority for an \gls{order} that meets the following requirements:

\begin{enumerate}
  \item has an Available quantity greater than or equal to the \gls{quote} request quantity;
  \item does not have the Pending-Cancel flag set.
\end{enumerate}

If such an \gls{order} exists, then the Available quantity is reduced on the \gls{order} by the
quantity specified on the \gls{quote} request, and that side of the \gls{quote} is set to the
\gls{order}['s] \gls{limit price}. A \gls{quote} is sent back to the customer if an \gls{order} is
found for at least one side of the \gls{market}. Otherwise, a \gls{quote} response is sent back to
the customer indicating a lack of available \gls{liquidity}.

\begin{appendices}

\chapter{API Specification}

\section{Order Management}

\subsection{Place Order}

\vspace{5mm}
\begin{tabular}{ll|ll|ll}
\multicolumn{2}{c}{Rest API}&\multicolumn{2}{c}{Fix API}\\
Name&Type&Tag&Field&Req'd&Description\\
\hline
Market&String&55&Symbol&Y&Market mnemonic\\
Ref&String&11&ClOrdId&Y&Client reference\\
Action&Enum&54&Side&Y&``Buy'' or ``Sell''\\
-&-&40&OrdType&Y&Always ``LIMIT''\\
Ticks&Long&44&Price&Y&Price in ticks\\
Lots&Long&38&OrderQty&Y&Quantity in lots\\
MinLots&Long&110&MinQty&Y&Minimum quantity in lots\\
-&-&60&TransactTime&Y&Current time\\
\end{tabular}
\vspace{5mm}

\subsection{Revise Order}

\vspace{5mm}
\begin{tabular}{ll|ll|ll}
\multicolumn{2}{c}{Rest API}&\multicolumn{2}{c}{Fix API}\\
Name&Type&Tag&Field&Req'd&Description\\
\hline
Market&String&55&Symbol&Y&Market mnemonic\\
-&-&11&ClOrdId&Y&Client reference\\
Id&Long&37&OrderId&Y&Order identifier\\
-&-&41&OrigClOrdId&Y&Original client reference\\
-&-&54&Side&Y&Always ``Undisclosed''\\
-&-&40&OrdType&Y&Always ``LIMIT''\\
Lots&Long&38&OrderQty&Y&Quantity in lots\\
-&-&60&TransactTime&Y&Current time\\
\end{tabular}
\vspace{5mm}

\subsection{Cancel Order}

\vspace{5mm}
\begin{tabular}{ll|ll|ll}
\multicolumn{2}{c}{Rest API}&\multicolumn{2}{c}{Fix API}\\
Name&Type&Tag&Field&Req'd&Description\\
\hline
Market&String&55&Symbol&Y&Market mnemonic\\
-&-&11&ClOrdId&Y&Client reference\\
Id&Long&37&OrderId&Y&Order identifier\\
-&-&41&OrigClOrdId&Y&Original client reference\\
-&-&54&Side&Y&Always ``Undisclosed''\\
-&-&38&OrderQty&Y&Always zero. Ignored by server-side\\
-&-&60&TransactTime&Y&Current time\\
\end{tabular}
\vspace{5mm}

\subsection{Order Execution}

\vspace{5mm}
\begin{tabular}{ll|ll|ll}
\multicolumn{2}{c}{Rest API}&\multicolumn{2}{c}{Fix API}\\
Name&Type&Tag&Field&Req'd&Description\\
\hline
Id&Long&17&ExecId&Y&-\\
OrderId&Long&37&OrderId&Y&-\\
Trader&String&1&Account&Y&-\\
Market&String&55&Symbol&Y&-\\
Contr&String&20000&Contract&Y&-\\
SettlDay&Int&64&FutSettDate&Y&-\\
Ref&String&11&ClOrdId&Y&-\\
State&State&150&ExecType&Y&-\\
-&-&39&OrdStatus&Y&-\\
Action&Enum&54&Side&Y&``Buy'' or ``Sell''\\
-&-&40&OrdType&Y&Always ``LIMIT''\\
Ticks&Long&44&Price&Y&Price in ticks\\
Lots&Long&38&OrderQty&Y&Quantity in lots\\
Resd&Long&151&LeavesQty&Y&-\\
Exec&Long&14&CumQty&Y&-\\
Cost&Long&20001&Cost&Y&-\\
-&-&6&AvgPx&Y&-\\
LastTicks&Long&31&LastPx&C&-\\
LastLots&Long&32&LastQty&C&-\\
MinLots&Long&110&MinQty&Y&-\\
MatchId&Long&20002&MatchId&C&-\\
Role&Role&851&LastLiquidityInd&C&-\\
Cpty&String&375&ContraBroker&C&-\\
Created&Long&60&TransactTime&Y&-\\
\end{tabular}
\vspace{5mm}

\chapter{Conversions}

\section{Rounding}

The \texttt{roundHalfAway} function is defined as follows:

\[r(x) = \left\{\begin{array}{ll}
  0 & \mbox{if $x = 0$};\\
  \ceil{x-0.5} & \mbox{if $x < 0$};\\
  \floor{x+0.5} & \mbox{if $x > 0$}.\\
\end{array}\right.
\]

\end{appendices}

\newpage
\bibliographystyle{plain}
\bibliography{swirly}
\printglossaries

\end{document}
