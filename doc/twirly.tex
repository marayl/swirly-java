\documentclass[11pt,a4paper]{article}

\usepackage{hyperref}
\hypersetup{
  bookmarks=true,
  pdftitle={The Twirly Manual},
  pdfauthor={Mark Aylett},
  colorlinks=true,
  linkcolor=blue,
  citecolor=blue,
  filecolor=blue,
  urlcolor=blue
}

\usepackage{float}
\usepackage{graphicx}
\usepackage{listings}
\usepackage[titletoc,title,toc]{appendix}
\usepackage[acronym,toc,xindy]{glossaries}
\makeglossaries

\newcommand{\ceil}[1]{\left\lceil #1 \right\rceil}
\newcommand{\floor}[1]{\left\lfloor #1 \right\rfloor}

\newcommand{\scltd}{Swirly Cloud Limited\cite{swirlycloud:www}}
\newcommand{\scorg}{Swirly Cloud\cite{swirlycloud:github}}
\newcommand{\twirly}{Twirly\cite{twirly}}

\newacronym{api}{API}{Application Programming Interface}
\newacronym{bbo}{BBO}{Best Bid and Offer}
\newacronym{cpu}{CPU}{Central Processing Unit}
\newacronym{ecn}{ECN}{Electronic Communication Network}
\newacronym{json}{JSON}{Java-Script Object Notation}
\newacronym{hft}{HFT}{High-Frequency Trading}
\newacronym{http}{HTTP}{Hyper-Text Transfer Protocol}
\newacronym{ip}{IP}{Intellectual Property} 
\newacronym{mtm}{MTM}{Mark To Market}
\newacronym{poc}{POC}{Proof Of Concept}
\newacronym{rest}{Rest}{Representational State Transfer}
\newacronym{tco}{TCO}{Total Cost of Ownership}
\newacronym{tob}{TOB}{Top Of Book}
\newacronym{ui}{UI}{User Interface}
\newacronym{vwap}{VWAP}{Volume Weighted Average Price}
\newacronym{www}{WWW}{World Wide Web}

\newglossaryentry{appengine}{name={App Engine}, description={application services running on
    Google's Cloud platform}}

\newglossaryentry{arbitrage}{name={arbitrage}, description={the simultaneous purchase and sale on
    different markets, of the same or equivalent financial instruments to profit from price or
    currency differentials}}

\newglossaryentry{asset}{name={asset}, description={an item of value}}

\newglossaryentry{auction}{name={auction}, description={a market where goods or services are sold to
    the highest bidder}}

\newglossaryentry{bid}{name={bid}, description={the price and quantity at which a market participant
    is willing to buy}}

\newglossaryentry{bitcoin}{name={bitcoin}, description={an electronic currency that allows users to
    transact online without a financial intermediary}}

\newglossaryentry{cloud}{name={Cloud}, description={a network of remote servers hosted on the
    Internet to store, manage, and process data}}

\newglossaryentry{contract}{name={contract}, description={a specification that stipulates the terms
    and conditions of sale}}

\newglossaryentry{exchange}{name={exchange}, description={a place where buyers and sellers meet to
    exchange goods or services}}

\newglossaryentry{execution}{name={execution}, description={a transaction that occurs as an order
    transitions through a workflow}}

\newglossaryentry{expiry date}{name={expiry date}, description={the last date that a contract can be
    traded on a specific market}}

\newglossaryentry{given}{name={given}, description={when the taker hits the bid}}

\newglossaryentry{price-level}{name={price-level}, description={a price level is the sum of all
    orders in the book with the same price}}

\newglossaryentry{liquid market}{name={liquid market}, description={the ability of a market to
    accept large transactions with minimal impact on price stability}}

\newglossaryentry{liquidity}{name={liquidity}, description={refers the bids and offers available in
    a market}}

\newglossaryentry{lot}{name={lot}, description={a unit of measure that represents the smallest
    amount that can be bought or sold}}

\newglossaryentry{maker}{name={maker}, description={passive buyer or seller that receives the spread}}

\newglossaryentry{market}{name={market}, description={a place where buyers and sellers come together
    to exchange goods or services}}

\newglossaryentry{market-maker}{name={market-maker}, description={a market-maker quotes both bids
    and offers, aiming to profit from the bid-offer spread}}

\newglossaryentry{market order}{name={market order}, description={an order to buy or sell at the
    current market price}}

\newglossaryentry{market-place}{name={market-place}, description={a venue comprising many markets}}

\newglossaryentry{matching engine}{name={matching engine}, description={the software component in a
    trading application responsible for matching buy and sell orders to form a trade}}

\newglossaryentry{microservice}{name={microservice}, description={a small, independent and
    language-agnostic service that does one thing and does it well}}

\newglossaryentry{order-book}{name={order-book}, description={the software component in a trading
    application responsible for managing resting orders in a two-sided market}}

\newglossaryentry{offer}{name={offer}, description={the price and quantity at which a market
    partipant is willing to sell}}

\newglossaryentry{order}{name={order}, description={an instruction to buy or sell goods or
    services}}

\newglossaryentry{paid}{name={paid}, description={when the taker lifts the offer}}

\newglossaryentry{servlet}{name={servlet}, description={a software component in a Web server that
    responds to client requests}}

\newglossaryentry{restful}{name={restful}, description={a software architecture for building
    scalable web services}}

\newglossaryentry{resting-order}{name={resting-order}, description={an order that is sitting in the
    order-book waiting to be matched}}

\newglossaryentry{settlement date}{name={settlement date}, description={the date on which a trade is
    settled between counter-parties}}

\newglossaryentry{spread}{name={spread}, description={the difference between the best bid and
    offer}}

\newglossaryentry{taker}{name={taker}, description={aggressive buyer or seller that pays the spread}}

\newglossaryentry{tick}{name={tick}, description={a unit of measure that represents the smallest
    price movement}}

\newglossaryentry{trade}{name={trade}, description={the exchange of goods or services between
    counter-parties}}

\newglossaryentry{trade date}{name={trade date}, description={the date on which a trade takes place}}

\newglossaryentry{unzip}{name={unzip}, description={a program used to extract compressed files in a
    ZIP archive}}

\newglossaryentry{web}{name={Web}, description={the World Wide Web}}

\glsaddall

\begin{document}

\title{The Twirly Manual}
\author{Mark Aylett}
\date{\today\\Version 0.1}
\maketitle

\tableofcontents

\section{Introduction}

\Glspl{order-book} form the backbone of many trading applications used by \glspl{exchange},
\glspl{ecn}, hedge funds and investment banks. They are used by several classes of trading
application within these organisations, including, but not limited to, the following:

\begin{itemize}
\item order management systems;
\item liquidity aggregators;
\item internalisation engines;
\item dark pools.
\end{itemize}

The \glspl{order-book} themselves are often specific to an \gls{asset}-class, and tightly coupled
with surrounding componentry and business logic. I have personally worked on several such systems,
and I was often dissapointed by the lack of generality and missed opportunity for reuse, especially
after repeatedly solving the same problem for many organisations.

Some, of course, will claim that their particular \gls{order-book} is superior to the rest, and that
its implementation must remain a closely guarded secret to protect their \gls{ip} and maintain a
competitive edge. While there may be some truth to this argument -- there are good and bad ways to
implement an \gls{order-book} -- a trading application's \gls{ip} is likely to be dominated by other
domain-specific technological and business innovations, and it makes more sense to focus one's
investment in these areas.

\Glspl{order-book} are by no means the reserve of large financial organisations. We shall see in the
following section that, given the right level of abstraction, they can be deployed with great effect
to solve broad range of applications from online \glspl{auction}, to price comparison websites, and
even \gls{bitcoin} \glspl{exchange}.

\section{Key Concepts}

By way of introduction, I will attempt to explain how an \gls{order-book} works in terms of more
familiar concepts. If you already know what an \gls{order-book} is, and you familiar with the
nomenclature, then feel free to skim this section and move on to the next.

We'll start with something that we're all familiar with: purchasing some goods from a
retailer. Let's say that an online retailer is offering to sell milk at $49$ pence per pint, and
they only have 10 pints in stock. This can be depicted as follows:

\vspace{5mm}
\begin{tabular}{rr}
\multicolumn{2}{c}{Milk price per pint}\\
Offer Price&Offer Lots\\
\hline
\texttt{0.49}&\texttt{10}\\
\end{tabular}
\vspace{5mm}

(In this introduction, we'll assume that \glspl{lot} means quantity, so 10 \glspl{lot} equals 10
pints in this example.)

If we decide to buy 3 of the 10 pints on \gls{offer}, then only 7 will remain. The online retailer
may, rather unscrupulously, decide to raise the price if they believe that supply outstrips demand:

\vspace{5mm}
\begin{tabular}{rr}
\multicolumn{2}{c}{Milk price per pint}\\
Offer Price&Offer Lots\\
\hline
\texttt{0.51}&\texttt{7}\\
\end{tabular}
\vspace{5mm}

This price is presumably non-negotiable, so we have to either ``take'' it or leave it. But what if
we could negotiate, or ``\gls{bid}'' for the remaining \glspl{lot}? You would then have a system
similar to an online ``\gls{auction}'', such as eBay\cite{ebay}, where the \gls{offer} of $51$ pence
is effectively the ``Buy Now'' price. Let's say we place a \gls{bid} to buy 3 more pints of milk at
$48$ pence:

\vspace{5mm}
\begin{tabular}{rrrrr}
\multicolumn{5}{c}{Milk price per pint}\\
Bid Lots&Bid Price&Spread&Offer Price&Offer Lots\\
\hline
\texttt{3}&\texttt{0.48}&\texttt{0.03}&\texttt{0.51}&\texttt{7}\\
\end{tabular}
\vspace{5mm}

The difference between the \gls{bid} and \gls{offer} price is $3$ pence, which is known as the
``\gls{spread}''. In an online \gls{auction}, another buyer may compete by raising the ``best''
\gls{bid} to ``buy 2 \glspl{lot} at $49$'':

\vspace{5mm}
\begin{tabular}{r|rrrrr}
\multicolumn{6}{c}{Milk price per pint}\\
Level&Bid Lots&Bid Price&Spread&Offer Price&Offer Lots\\
\hline
\texttt{1}&\texttt{2}&\texttt{0.49}&\texttt{0.02}&\texttt{0.51}&\texttt{7}\\
\texttt{2}&\texttt{3}&\texttt{0.48}&\texttt{-}&\texttt{-}&\texttt{-}\\
\end{tabular}
\vspace{5mm}

There are now two ``\glspl{price-level}'' on the \gls{bid}-side of the ``\gls{order-book}''. Our
order is said to be ``resting'' at the second \gls{price-level}, beneath the \gls{tob}. The
\gls{spread} on the first \gls{price-level} has now been reduced to $2$ pence. (There is no
\gls{spread} on the second level, because there is no \gls{offer} at this level.)

Note also at this point that the average price the seller can expect to receive for 5 of the 7
\glspl{lot} they are offering is $48.4$ pence:

\[
\frac{1}{n}\sum_{i=1}^{n}p_iq_i = \frac{2\times0.49+3\times0.48}{5} = 48.4 \enskip\mbox{pence}
\]

Where $p$ is price, $q$ is \glspl{lot}, and $n$ is total \glspl{lot} (or ``volume''). This is known
as the \gls{vwap}.

If this new buyer then decides to ``revise'' their \gls{bid} to $51$ pence, then the \gls{bid} will
``cross the \gls{spread}'' and match the \gls{offer} to create a trade, which will further reduce
the supply down to 5 \glspl{lot}. The \gls{offer} of $48$ pence will once again be the \gls{tob}:

\vspace{5mm}
\begin{tabular}{r|rrrrr}
\multicolumn{6}{c}{Milk price per pint}\\
Level&Bid Lots&Bid Price&Spread&Offer Price&Offer Lots\\
\hline
\texttt{1}&\texttt{3}&\texttt{0.48}&\texttt{0.03}&\texttt{0.51}&\texttt{5}\\
\texttt{2}&\texttt{-}&\texttt{-}&\texttt{-}&\texttt{-}&\texttt{-}\\
\end{tabular}
\vspace{5mm}

All manner of goods can be sold in an online \gls{auction}. Oftentimes, the items on sale are one of
a kind. If would be difficult, for example, to \gls{offer} many second-hand bicycles of the exact
same specification, so each has to be sold separately.

It is interesting to contrast this idea of an online \gls{auction} with a price-comparison
website. Perhaps, as prospective buyers, we're interested in comparing prices for the latest smart
phone, offered by many suppliers:

\vspace{5mm}
\begin{tabular}{r|rrrrr}
\multicolumn{6}{c}{Smart phone prices}\\
Level&Bid Lots&Bid Price&Spread&Offer Price&Offer Lots\\
\hline
\texttt{1}&\texttt{-}&\texttt{-}&\texttt{-}&\texttt{419.99}&\texttt{5}\\
\texttt{2}&\texttt{-}&\texttt{-}&\texttt{-}&\texttt{429.99}&\texttt{12}\\
\texttt{3}&\texttt{-}&\texttt{-}&\texttt{-}&\texttt{449.99}&\texttt{20}\\
\end{tabular}
\vspace{5mm}

Here there are 3 \glspl{price-level} in the \gls{order-book}, and we can see that there is a greater
quantity on \gls{offer} if we're willing to pay more. (This is unsurprising given that Traders like
to secure a profit by buying low and selling high!)  We could also compare milk prices from many
suppliers in the same way, but we'd first need to define exactly what kind of milk we were selling,
in what quantity, and when it would be delivered. This information would form the basis of a
``\gls{contract}'':

\vspace{5mm}
\begin{tabular}{r|rrrrr}
\multicolumn{6}{c}{UK Farmgate Milk prices per litre}\\
\multicolumn{6}{c}{December 2014}\\
Level&Bid Lots&Bid Price&Spread&Offer Price&Offer Lots\\
\hline
\texttt{1}&\texttt{-}&\texttt{-}&\texttt{-}&\texttt{27.85}&\texttt{50}\\
\texttt{2}&\texttt{-}&\texttt{-}&\texttt{-}&\texttt{27.86}&\texttt{120}\\
\texttt{3}&\texttt{-}&\texttt{-}&\texttt{-}&\texttt{27.88}&\texttt{200}\\
\end{tabular}
\vspace{5mm}

What if we were now to combine the online \gls{auction} with the price comparison website? We would
then have something much more reminiscent of a wholesale \gls{market} in an ``\gls{exchange}'',
where there are many buyers and sellers:

\vspace{5mm}
\begin{tabular}{r|rrrrr}
\multicolumn{6}{c}{UK Farmgate Milk prices per litre}\\
\multicolumn{6}{c}{December 2014}\\
Level&Bid Lots&Bid Price&Spread&Offer Price&Offer Lots\\
\hline
\texttt{1}&\texttt{45}&\texttt{27.83}&\texttt{0.02}&\texttt{27.85}&\texttt{50}\\
\texttt{2}&\texttt{135}&\texttt{27.81}&\texttt{0.05}&\texttt{27.86}&\texttt{120}\\
\texttt{3}&\texttt{-}&\texttt{-}&\texttt{-}&\texttt{27.88}&\texttt{200}\\
\end{tabular}
\vspace{5mm}

Market participants would typically enter orders electronically via \gls{ui} similar to the
following:

\begin{figure}[H]
\centering
\includegraphics[scale=0.6]{milkorder.eps}
\caption{Placing a milk order}
\end{figure}

A larger number of \gls{market} participants generally helps to improve the \gls{market}'s
efficiency and stability. This is known as a ``\gls{liquid market}''.

In this section, we've seen how the concept of an \gls{order-book} can be used to express different
kinds of \gls{market-place}, including online \glspl{auction}, price comparison websites, and
wholesale \glspl{market}. \Glspl{order-book} are prevalent in financial organisations, where they
are used to trade financial products across many different \gls{asset}-classes, including
commodities, corporate and government bonds, currencies, equities and indexes.

\section{Architecture}

\twirly{} is a \gls{microservice} that provides a \gls{restful} \gls{order-book} \gls{api} to for
querying reference-data, market-data, and managing \glspl{order}. Twirly's\cite{twirly} software
architecture is designed to achieve the following goals:

\begin{itemize}
\item clean and simple user experience;
\item mobile, \gls{web} and desktop equality;
\item language- and platform-agnostic \gls{api};
\item scalable \gls{microservice} architecture;
\item elastic storage and compute;
\item efficient resource consumption;
\item highly available, reliable and secure;
\item low maintenance and low \gls{tco};
\item use of open standards and technologies.
\end{itemize}

A \Gls{restful} \gls{web} architecture was chosen because it is scalable and equally well suited to
mobile, \gls{web} and desktop clients. Its generality makes it suitable for retail and wholesale
trading applications, across \gls{asset}-class and product-type. It can be deployed to the
\gls{cloud} or on dedicated infrastructure.

The following diagram shows how the facets of the \gls{restful} \gls{api} map to internal
\glspl{servlet}, along with the \gls{http} methods and resources supported by those \glspl{servlet}:

\begin{figure}[H]
\centering
\includegraphics[scale=0.6]{serv.eps}
\caption{Restful API}
\end{figure}

The project was originally started as a \gls{poc} for a low-latency \gls{matching engine}. \scltd{}
was formed to take ownership of the project and put the company on a commercial footing. The current
development phase is being funded by revenue generated from consultancy by \scltd{}.

\subsection{Reference Data}

The following table summarises the \gls{restful} \gls{api} for querying reference-data:

\vspace{5mm}
\begin{tabular}{ll}
\multicolumn{2}{c}{Synopsis}\\
Method&Resource\\
\hline
GET&\texttt{/api/rec/}\\
GET&\texttt{/api/rec/asset/[mnem]}\\
GET&\texttt{/api/rec/contr/[mnem]}\\
GET&\texttt{/api/rec/market/[mnem]}\\
GET&\texttt{/api/rec/trader/[mnem]}\\
\end{tabular}
\vspace{5mm}

\begin{figure}[H]
\centering
\includegraphics[scale=0.6]{rec.eps}
\caption{Record Structures}
\end{figure}

\subsection{Market Data}

The following table summarises the \gls{restful} \gls{api} for querying market-data:

\vspace{5mm}
\begin{tabular}{ll}
\multicolumn{2}{c}{Synopsis}\\
Method&Resource\\
\hline
GET&\texttt{/api/view/[market]}\\
\end{tabular}
\vspace{5mm}

\begin{figure}[H]
\centering
\includegraphics[scale=0.6]{view.eps}
\caption{View Structures}
\end{figure}

\subsection{Order Management}

\vspace{5mm}
\begin{tabular}{ll|l}
\multicolumn{2}{c}{Rest API}&\multicolumn{1}{c}{Fix API}\\
Method&Resource&Message\\
\hline
GET&\texttt{/api/sess/}\\
GET&\texttt{/api/sess/order/[market]/[id]}\\
GET&\texttt{/api/sess/trade/[market]/[id]}\\
GET&\texttt{/api/sess/posn/[contr]/[settlDate]}\\
DELETE&\texttt{/api/sess/order/<market>/<id>}\\
DELETE&\texttt{/api/sess/trade/<market>/<id>}\\
POST&\texttt{/api/sess/order/<market>/}&NewOrderSingle\\
PUT&\texttt{/api/sess/order/<market>/<id>}\\
\end{tabular}
\vspace{5mm}

\subsubsection{Executions}

\vspace{5mm}
\begin{tabular}{ll|ll|ll}
\multicolumn{2}{c}{Rest API}&\multicolumn{2}{c}{Fix API}\\
Name&Type&Tag&Field&Req'd&Description\\
\hline
Id&Long&17&ExecId&Y&-\\
OrderId&Long&37&OrderId&Y&-\\
Trader&String&1&Account&Y&-\\
Market&String&55&Symbol&Y&-\\
Contr&String&20000&Contract&Y&-\\
SettlDay&Int&64&FutSettDate&Y&-\\
Ref&String&11&ClOrdId&Y&-\\
State&State&150&ExecType&Y&-\\
-&-&39&OrdStatus&Y&-\\
Action&Enum&54&Side&Y&``Buy'' or ``Sell''\\
-&-&40&OrdType&Y&Always ``LIMIT''\\
Ticks&Long&44&Price&Y&Price in ticks\\
Lots&Long&38&OrderQty&Y&Quantity in lots\\
Resd&Long&151&LeavesQty&Y&-\\
Exec&Long&14&CumQty&Y&-\\
Cost&Long&20001&Cost&Y&-\\
-&-&6&AvgPx&Y&-\\
LastTicks&Long&31&LastPx&C&-\\
LastLots&Long&32&LastQty&C&-\\
MinLots&Long&110&MinQty&Y&-\\
MatchId&Long&20002&MatchId&C&-\\
Role&Role&851&LastLiquidityInd&C&-\\
Cpty&String&375&ContraBroker&C&-\\
Created&Long&60&TransactTime&Y&-\\
\end{tabular}
\vspace{5mm}

\begin{figure}[H]
\centering
\includegraphics[scale=0.6]{instruct.eps}
\caption{Order Structures}
\end{figure}

\begin{figure}[H]
\centering
\includegraphics[scale=0.6]{state.eps}
\caption{Order Workflow}
\end{figure}

\subsubsection{Order Placement}

\vspace{5mm}
\begin{tabular}{ll|ll|ll}
\multicolumn{2}{c}{Rest API}&\multicolumn{2}{c}{Fix API}\\
Name&Type&Tag&Field&Req'd&Description\\
\hline
Market&String&55&Symbol&Y&Market mnemonic\\
Ref&String&11&ClOrdId&Y&Client reference\\
Action&Enum&54&Side&Y&``Buy'' or ``Sell''\\
-&-&40&OrdType&Y&Always ``LIMIT''\\
Ticks&Long&44&Price&Y&Price in ticks\\
Lots&Long&38&OrderQty&Y&Quantity in lots\\
MinLots&Long&110&MinQty&Y&Minimum quantity in lots\\
-&-&60&TransactTime&Y&Current time\\
\end{tabular}
\vspace{5mm}

\begin{figure}[H]
\centering
\includegraphics[scale=0.6]{neworder.eps}
\caption{New Order Dialog}
\end{figure}

\subsubsection{Order Revision}

\begin{figure}[H]
\centering
\includegraphics[scale=0.6]{reviseorder.eps}
\caption{Revise Order Dialog}
\end{figure}

\subsection{Administration}

\vspace{5mm}
\begin{tabular}{ll}
\multicolumn{2}{c}{Synopsis}\\
Method&Resource\\
\hline
POST&\texttt{/api/rec/market/}\\
POST&\texttt{/api/rec/trader/}\\
\end{tabular}
\vspace{5mm}

\begin{figure}[H]
\centering
\includegraphics[scale=0.6]{newmarket.eps}
\caption{New Market Dialog}
\end{figure}

To create a new \gls{market}, submit an \gls{http} post request to the \texttt{/api/rec/market/}
resource with the following \gls{json} content:

{\tt\small
\begin{verbatim}
{"mnem":"EURUSD.MAR15","display":"EURUSD March 15","contr":"EURUSD",
 "settlDate":20150320,"expiryDate":20150318}
\end{verbatim}
}

\begin{figure}[H]
\centering
\includegraphics[scale=0.6]{newtrader.eps}
\caption{New Trader Dialog}
\end{figure}

To create a new trader, submit an \gls{http} post request to the \texttt{/api/rec/trader/} resource
with the following \gls{json} content:

{\tt\small
\begin{verbatim}
{"mnem":"EURUSD.MAR15","display":"EURUSD March 15",
 "email":"mark.aylett@swirlycloud.com"}
\end{verbatim}
}

\section{Swirly Cloud Demo}

\subsection{Cloud Computing}

We believe that cost-cutting by financial organisations will lead to new software architectures that
can dynamically grow or shrink to meet demand.

Large online retailers and tech companies have invested heavily in technologies, such as
Google's\cite{google} BigTable\cite{bigtable}, that can cope with Internet-scale workloads, and they
are providing access to their global storage and compute as a managed service (the \gls{cloud}). The
landscape of modern software development has fundamentally changed as a consequence.

The beauty of \gls{cloud} computing is that we only pay for what we use, so we don't need to invest
our capital on infrastructure and additional capacity for future growth. The \gls{cloud} also
simplifies maintenance, operations, disaster recovery and security, so that we can focus our efforts
on writing great software.

\twirly{} has been designed from the ground-up to run efficiently in the \gls{cloud} with limited
resources.  It draws on lessons learned from years spent building front-office trading applications,
and adapts their best ideas for the \gls{cloud}. The \gls{order-book} at the heart of our
application has been developed using \gls{hft} techniques, so that we can achieve microsecond
latencies.

We deliberately develop and test our system with very limited resource quotas, so that we remain
focused on our bottom line. We have also taken care to avoid vendor lock-in, by ensuring that our
core application only uses open standards and technologies. This gives us the option to target
different application containers, run on dedicated infrastructure, or to open-source the software.

\subsection{User Experience}

When I download a software package for the first time, I \gls{unzip} it and attempt to compile it
using familiar commands, without even a fleeting glimpse at the ``INSTALL'' or ``README''
documents. If the build fails, then I discard it and move onto the next, mentally branding it as
``broken'' or ``too complex''. Similarly, when I try a new website for the first time and I'm
presented with a complex, lengthy and intrusive sign-up processes, or I find the site cluttered and
confusing, or awash with advertising, then I throw my hands up in despair and begin my search anew
for an alternative.

Am I being unfair? Probably, but then I'm not the only one. If the Internet generation is guilty of
a short attention span, then the Facebook\cite{facebook} generation is even more so. Often for good
reason. There simply isn't enough time in the day. Most of us are overloaded and stressed-out trying
to manage our busy lives. We mine and sift through mountains of information every day. We have to
make every second of every minute count. Human cycles are precious and most certainly more so than
\gls{cpu} cycles!

We were acutely aware of these factors when designing our \twirly{} \gls{ui}. We could not afford to
drop the ball when privileged with those few valuable moments in which to attract a new user. Our
barrier to entry had to be kept low.

We put new users on a fast track that allows them to get started quickly. They can sign-in using
their Google\cite{google} account, so there is no need to create and maintain yet another set of
online credentials. Any non-essential information has been removed from the sign-up process and
deferred to a preferences screen. The \gls{ui} is free of clutter and unnecessary
distractions. There are no gimmicks, pop-ups or excessive use of colour. The layout and workflows
follow the principle of least surprise, so that they feel familiar and intuitive to professional
traders. We work hard to retain a simple, consistent and functional interface that works well on a
range of devices from mobile to desktop.

\begin{appendices}

\section{Contributing}

The \twirly{} project is available on GitHub\cite{github} for members of the \scorg{} organisation.

\subsection{Getting Started}

\scorg{} members can clone the \twirly{} repository as follows:

\lstset{language=sh}
\begin{lstlisting}
$ git clone git@github.com:swirlycloud/twirlyj.git 
\end{lstlisting}

Developers should, however, work on their own fork of the repository and submit pull requests to
have their patches reviewed and possibly merged onto the \texttt{master} branch.

Once you have a local copy of the repository checked-out, add a remote upstream repository so that
you can keep your fork in sync:

\lstset{language=sh}
\begin{lstlisting}
$ git remote add upstream git@github.com:swirlycloud/twirlyj.git
\end{lstlisting}

You can build the source code and run the unit tests using Gradle\cite{gradle} as follows:

\lstset{language=sh}
\begin{lstlisting}
$ gradle build
\end{lstlisting}

The application can then be tested locally by running the following command and browsing to
\url{http://localhost:8080/}:

\lstset{language=sh}
\begin{lstlisting}
$ gradle appengineRun
\end{lstlisting}

Eclipse\cite{eclipse} project files can also be generated using the following command:

\lstset{language=sh}
\begin{lstlisting}
$ gradle eclipse
\end{lstlisting}

\subsection{Syncing Your Fork}

You'll first need to fetch branches and their commits from the upstream repository:

\lstset{language=sh}
\begin{lstlisting}
$ git fetch upstream
\end{lstlisting}

Commits from the upstream's \texttt{master} branch will now be stored in a local branch called
\texttt{upstream/master}. Check out your fork's local \texttt{master} branch and rebase from the
\texttt{upstream/master} branch:

\lstset{language=sh}
\begin{lstlisting}
$ git checkout master
$ git rebase upstream/master
\end{lstlisting}

Your local fork's \texttt{master} branch should now be in sync with the upstream
repository. Finally, you can push your local \texttt{master} branch to origin \texttt{master}, so
that origin \texttt{master} is also in sync with the upstream repository:

\lstset{language=sh}
\begin{lstlisting}
$ git push origin master
\end{lstlisting}

\section{Conversions}

\subsection{Rounding}

The \texttt{roundHalfAway} function is defined as follows:

\[r(x) = \left\{\begin{array}{ll}
  0 & \mbox{if $x = 0$};\\
  \ceil{x-0.5} & \mbox{if $x < 0$};\\
  \floor{x+0.5} & \mbox{if $x > 0$}.\\
\end{array}\right.
\]

\end{appendices}

\newpage
\addcontentsline{toc}{section}{References}
\bibliographystyle{plain}
\bibliography{twirly}
\printglossaries

\end{document}
